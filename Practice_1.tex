\documentclass[12pt]{article}
\parindent=0cm

\usepackage[T2A]{fontenc} % Используем 8-битные шрифты, которые поддерживают кириллицу
\usepackage[utf8]{inputenc} % Кодировка файла
\usepackage[english, russian]{babel} % Пакет логики переносов

\newcommand{\nl}{\vspace{\baselineskip}} % Команда создания пустой строки

\begin{document}
{\bf Проба пера.}

\begin{center}
Все строки этого абзаца будут
центрированы; переносов не будет,
если только какое-то слово,
как в дезоксирибонуклеиновой
кислоте, не длинней строки.
\end{center}

{\bf Текст. Формулы.}\nl

Катеты $a$, $b$ треугольника связаны с его гипотенузой $c$ формулой ${c^2=a^2+b^2}$ (теорема Пифагора).
Из теоремы Ферма следует, что уравнение $$x^{4357}+y^{4357} = z^{4357}$$ не имеет решений в натуральных числах.\nl

Обозначение $R^i_{jkl}$ для тензора кривизны было введено еще Эйнштейном.
Можно также написать $R_j{}^i{}_{kl}$ , хотя не всем это нравится.\nl

Неравенство $x+1/x\ge2$ выполнено для всех $x>0$.
$\pi\approx3{,}14$
$$\frac{(a+b)^2}{4}-\frac{(a-b)^2}{4}=ab$$
$$\frac 12 +\frac x 2 = \frac{1+x}2$$
$$1+\left(\frac1{1-x^2}\right)^3$$\nl

По общепринятому соглашению, $\sqrt[3]{x^3}=x$, но $\sqrt{x^2}=|x|$.
Согласно формуле Лейбница, $$(fg)''= f''g + 2f'g' + fg'' .$$ Это похоже на формулу квадрата суммы: ${x'}^2$.\nl

В детстве К.-Ф.Гаусс придумал, как быстро найти сумму $$1+2+\cdots+100=5050;$$ это случилось, когда школьный учитель задал классу найти сумму чисел $1,2,\ldots,100$.
\end{document}