\documentclass[12pt]{article}
\parindent=0cm

\usepackage[T2A]{fontenc} % Используем 8-битные шрифты, которые поддерживают кириллицу
\usepackage[utf8]{inputenc} % Кодировка файла
\usepackage[english, russian]{babel} % Пакет логики переносов
\usepackage{amsmath, amssymb}

\newcommand{\nl}{\vspace{\baselineskip}} % Команда создания пустой строки

\begin{document}
\begin{center} \bf{Домашнее задание по практике 4} \end{center}

$$
(-1)^p \det \textbf{C}
\begin{pmatrix}
i_1 &i_2 &\ldots & i_p \\
i_1 &i_2 &\ldots &i_p
\end{pmatrix}
 > 0, \;
\begin{aligned}
&1 \leq i_1 < i_2 < \ldots <  i_1 \leq  n;\\
&p = 1, 2, \ldots, n,
\end{aligned} \eqno (11)
$$

$$
\begin{pmatrix}
||\textbf{A}^{(-1)}_{11}|| &||\textbf{A}^{(-1)}_{12}|| \\
||\textbf{A}^{(-1)}_{21}|| &||\textbf{A}^{(-1)}_{22}||
\end{pmatrix}
\leq \frac{
\begin{pmatrix}
||\textbf{A}^{-1}_{22}||^{-1} &||\textbf{A}_{12}|| \\
||\textbf{A}^{-1}_{21}|| &||\textbf{A}^{-1}_{11}||^{-1} \\
\end{pmatrix}
}
{
||\textbf{A}^{-1}_{11}||^{-1}||\textbf{A}^{-1}_{22}||^{-1} - ||\textbf{A}_{12}||||\textbf{A}_{21}||
}
.
$$

\begin{equation*}
\begin{split}
spa \; \textbf{A} \leq \frac{||A_{11}||_{\log} + ||A_{22}||_{\log}}{2} + \\
+ \sqrt{\left( \frac{||A_{11}||_{\log} - ||A_{22}||_{\log}}{2} \right)^2 + ||A_{12}||||A_{21}|| }& < 0.
\end{split}
\end{equation*}

\begin{equation*}
\begin{split}
&||\textbf{A}||_0 = \max \limits_{1 \leq i \leq n} \sum \limits_{j = 1}^n |a_{ij}|,||\textbf{A}||_1 = \max \limits_{1 \leq i \leq n} \sum \limits_{j = 1}^n |a_{ij}|, \; ||\textbf{A}||_{1/2} = \sqrt{spa \textbf{A} * \textbf{A}}. \\
&||\textbf{A}||_{0 \log} =  \max \limits_{1 \leq i \leq n} \{\mathrm{Re} \; a_{ii} + p_i(\textbf{A})\}, \\
&||\textbf{A}||_{1 \log} =  \max \limits_{1 \leq i \leq n} \{\mathrm{Re} \; a_{ii} + q_i(\textbf{A})\}, \\
&\;\;\;\;\;\;||\textbf{A}||_{1/2 \log} = spa \frac{\textbf{A} + \textbf{A} ^*}{2} .
\end{split}
\end{equation*}

\begin{center}
\textit{Рядом с f\textup{(}x\textup{)} \textup{(}значение функции f в точке x\textup{)} лучше использовать прямые скобки (а не курсивные).}
\end{center}

\begin{center}
Когда одно из $_\text{слов}$ набрано шрифтом другого кегиля, это выглядит плохо.
\end{center}

\begin{center}
Мы зыкрываем группу и возвращаемся к обычному шрифту только после пустой строки, заверщающей абзац.

Вот шрифт обычного размера.

Здесь мы вернулись к обычному шрифту раньше времени, и межстрочные интервалы оказались слишком велики.

Вот шрифт обычного размера.
\end{center}

\begin{center}
Выберем \textbf{полужирный шрифт в \textit{курсивном} начертании} (временно, конечно же).
\end{center}
\end{document}