\documentclass{article}

\usepackage[T2A]{fontenc} % Используем 8-битные шрифты, которые поддерживают кириллицу
\usepackage[utf8]{inputenc} % Кодировка файла
\usepackage[english, russian]{babel} % Пакет логики переносов
\usepackage{amsmath, amssymb, amsfonts}

\newcommand{\nl}{\vspace{\baselineskip}} % Команда создания пустой строки

\begin{document}
УДК 517.984\nl

\begin{center}
\textbf{АБСОЛЮТНАЯ ЛОГАРИФМИЧЕСКАЯ НОРМА}\footnote{Работа выполнена при финансовой поддержке РФФИ, No 16-01-00197.}\nl

А. И. Иванов, О. И. Петров \nl

\textit{394006, Воронеж, Университетская пл., 1, Воронежский гос. ун-т.}
Поступила в редакцию **.11.20** г.\nl

\begin{minipage}{0.88\textwidth}
\;\,\, В статье вводится и изучается новое понятие – абсолютная логарифмическая норма, – которое имеет много общего с классическим определением логарифмической нормы, данным С. М. Лозинским.

\; \textbf{Ключевые слова:} логарифмическая норма, устойчивость по
Ляпунову, теория Перрона-Фробениуса неотрицательных матриц.
\end{minipage}
\end{center}

\begin{center} 1. ПРЕДВАРИТЕЛЬНЫЕ СВЕДЕНИЯ  \end{center}
Пусть $\textbf{A} = (a_{ij})$ – квадратная $n \times n$ - матрица и $\lambda_1 (\textbf{A}), \lambda_2 (\textbf{A}), \ldots , \lambda_m (\textbf{A})$ $(m \leq n)$ полный набор попарно различных собственных значений матрицы \textbf{A} (спектр sp \textbf{A}). Введём следующие числовые характеристики спектра

$$
spa \, \textbf{A} = \max \limits_{1 \leq i \leq m} \textup{Re} \, \lambda_i (\textbf{A}),\; spr \, \textbf{A} = \max \limits_{1 \leq i \leq m} | \lambda_i (\textbf{A})|. \eqno(1)
$$

\noindent Первая из формул (1.1) называется \textit{cпектральной абсциссой матрицы} \textbf{A}, авторая – хорошо известный её \textit{спектральный радиус}. Отметим, что всегда $|spa \, \textbf{A}| \leq spr \,  \textbf{A}.$

\begin{center} 2. АБСОЛЮТНАЯ ВЕЛИЧИНА (МОДУЛЬ) ВЕКТОРА \end{center}
Отметим простые свойства абсолютной величины вектора, напоминающие свойства нормы (сравни с [4, с. 127-129])

$1^0 \, |\textbf{x}| \geq \textbf{0},$

$2^0 \, |\textbf{x}| = \textbf{0} \iff \textbf{x} = \textbf{0},$

$3^0 \, |c \textbf{x}|  = |c||\textbf{x}| \, (c \in \mathbb{C}),$

$4^0 \, |\textbf{x} + \textbf{y}| \leq |\textbf{x}| + |\textbf{y}|$ (неравенство треугольника).

\noindent Из этих свойств вытекает обратное неравенство треугольника
$$
|\textbf{x} - \textbf{y}| \geq|  \, |\textbf{x}| + |\textbf{y}| \,  |. \eqno(2)
$$

\textbf{Лемма 1.} \textit{Для произвольных комплексных чисел x и h справедлива
формула}
$$
\lim \limits_{0 < t \to 0} \frac{|x + th| - |x|}{t} = \begin{cases}
\textup{Re} \, \frac{x \overline{h}}{|x|}, &\text{если} \, x \neq 0; \\
\; \; \; \; \; |h|, &\text{если} \, x = 0,
\end{cases} \eqno (3)
$$
\textit{\textup{(}где в верхней строке формулы \textup{(3)} черта – это черта комплексного сопряжения\textup{)}.}

\textbf{Теорема 1.} \textit{Дифференциал Гато $[\textbf{x}, \, \textbf{h}] : \mathbb{C}^n  \times \mathbb{C}^n \to \mathbb{R}^n$ обладает следующими свойствами:}

$1^0 \, |[\textbf{x}, \, \textbf{h}]| \leq |\textbf{h}|, \; [\textbf{0}, \, \textbf{h}]  = |\textbf{h}|;$

$2^0 \, [\textbf{x}, \, c\textbf{h}] = с[\textbf{x}, \textbf{h}]$ \,\,\,\, \, \textit{при} $c \geq 0,$

$\;\,\, \,\,\, [\textbf{x}, \, c\textbf{h}] = -с[\textbf{x}, -\textbf{h}]$ \textit{при} $c \geq 0,$

$3^0 \, [\textbf{x}, \textbf{h} + \textbf{k}] \leq [\textbf{x}, \, \textbf{h}] + [\textbf{x}, \, \textbf{k}] $ (\textit{полуаддитивность});

$4^0 \, [\textbf{x}, \textbf{h}] + [\textbf{x}, -\textbf{h}] \geq \textbf{0} \; (= [\textbf{x}, \textbf{0}]).$

По одной из формул получим
$$
[\textbf{e}_j, \, \textbf{Ae}_j] =
\begin{pmatrix}
|a_{1j}| \\
\ldots \\
\textup{Re} \, a_{jj} \\
\ldots \\
|a_{nj}| \\
\end{pmatrix}
\leq
\begin{pmatrix}
c_{1j} \\
\ldots \\
a_{jj} \\
\ldots \\
c_{nj} \\
\end{pmatrix} .
$$

$$
-\sum \limits_{k = 2}^\infty \frac{t^{k - 1}|\textbf{A}|^k}{k!} + \frac{|\textbf{I} + t\textbf{A}| - |\textbf{I}|}{t} | \leq \textbf{C} + \sum \limits_{k = 2}^\infty \frac{t^{k - 1} \textbf{C}^k}{k!}.
$$

$$
\dim \left( \sum \limits_{k = 1}^{m - 1} E_k \right) + \dim E_m = n.
$$

$$
\textup{im} \, \textbf{C} \! \dotplus \! \textup{ker} \, \textbf{C} = \mathbb{C}^n.
$$ \pagebreak

\begin{center}
СПИСОК ЛИТЕРАТУРЫ
\end{center}

\begin{enumerate}
\item \textit{Былов Б. Ф., Виноград Р. Э., Гробман Д. М., Немыцкий В. В.} Теория показателей Ляпунова и её приложения к вопросам устойчивости. – М. : Наука, 1966.
\item \textit{Лозинский С. М.} Оценка погрешности численного интегрирования обыкновенных дифференциальных уравнений // Изв. вузов. 1958. No 5. С. 52–90.
\end{enumerate}

\end{document}